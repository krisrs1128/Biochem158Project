\documentclass[12pt,english]{article}\usepackage{graphicx, color}
%% maxwidth is the original width if it is less than linewidth
%% otherwise use linewidth (to make sure the graphics do not exceed the margin)
\makeatletter
\def\maxwidth{ %
  \ifdim\Gin@nat@width>\linewidth
    \linewidth
  \else
    \Gin@nat@width
  \fi
}
\makeatother

\definecolor{fgcolor}{rgb}{0.2, 0.2, 0.2}
\newcommand{\hlnumber}[1]{\textcolor[rgb]{0,0,0}{#1}}%
\newcommand{\hlfunctioncall}[1]{\textcolor[rgb]{0.501960784313725,0,0.329411764705882}{\textbf{#1}}}%
\newcommand{\hlstring}[1]{\textcolor[rgb]{0.6,0.6,1}{#1}}%
\newcommand{\hlkeyword}[1]{\textcolor[rgb]{0,0,0}{\textbf{#1}}}%
\newcommand{\hlargument}[1]{\textcolor[rgb]{0.690196078431373,0.250980392156863,0.0196078431372549}{#1}}%
\newcommand{\hlcomment}[1]{\textcolor[rgb]{0.180392156862745,0.6,0.341176470588235}{#1}}%
\newcommand{\hlroxygencomment}[1]{\textcolor[rgb]{0.43921568627451,0.47843137254902,0.701960784313725}{#1}}%
\newcommand{\hlformalargs}[1]{\textcolor[rgb]{0.690196078431373,0.250980392156863,0.0196078431372549}{#1}}%
\newcommand{\hleqformalargs}[1]{\textcolor[rgb]{0.690196078431373,0.250980392156863,0.0196078431372549}{#1}}%
\newcommand{\hlassignement}[1]{\textcolor[rgb]{0,0,0}{\textbf{#1}}}%
\newcommand{\hlpackage}[1]{\textcolor[rgb]{0.588235294117647,0.709803921568627,0.145098039215686}{#1}}%
\newcommand{\hlslot}[1]{\textit{#1}}%
\newcommand{\hlsymbol}[1]{\textcolor[rgb]{0,0,0}{#1}}%
\newcommand{\hlprompt}[1]{\textcolor[rgb]{0.2,0.2,0.2}{#1}}%

\usepackage{framed}
\makeatletter
\newenvironment{kframe}{%
 \def\at@end@of@kframe{}%
 \ifinner\ifhmode%
  \def\at@end@of@kframe{\end{minipage}}%
  \begin{minipage}{\columnwidth}%
 \fi\fi%
 \def\FrameCommand##1{\hskip\@totalleftmargin \hskip-\fboxsep
 \colorbox{shadecolor}{##1}\hskip-\fboxsep
     % There is no \\@totalrightmargin, so:
     \hskip-\linewidth \hskip-\@totalleftmargin \hskip\columnwidth}%
 \MakeFramed {\advance\hsize-\width
   \@totalleftmargin\z@ \linewidth\hsize
   \@setminipage}}%
 {\par\unskip\endMakeFramed%
 \at@end@of@kframe}
\makeatother

\definecolor{shadecolor}{rgb}{.97, .97, .97}
\definecolor{messagecolor}{rgb}{0, 0, 0}
\definecolor{warningcolor}{rgb}{1, 0, 1}
\definecolor{errorcolor}{rgb}{1, 0, 0}
\newenvironment{knitrout}{}{} % an empty environment to be redefined in TeX

\usepackage{alltt}
\usepackage[T1]{fontenc}
\usepackage[latin9]{inputenc}
\usepackage{geometry}
\geometry{verbose}
\usepackage{amsthm}
\usepackage{amsmath}
\usepackage{esint}
\usepackage[colorlinks = true, citecolor = blue, linkcolor = magenta, urlcolor = green]{hyperref}
\usepackage{setspace}
\usepackage{fullpage}

\doublespace

\makeatletter
%%%%%%%%%%%%%%%%%%%%%%%%%%%%%% Textclass specific LaTeX commands.
\numberwithin{equation}{section}
\numberwithin{figure}{section}
\theoremstyle{plain}
\newtheorem{thm}{\protect\theoremname}
  \theoremstyle{remark}
  \newtheorem{claim}[thm]{\protect\claimname}

\makeatother

\usepackage{babel}
  \providecommand{\claimname}{Claim}
\providecommand{\theoremname}{Theorem}

\author{Kris Sankaran}
\date{\today}
\title{Biochem 158 Final Project}
\IfFileExists{upquote.sty}{\usepackage{upquote}}{}

\begin{document}

\maketitle

\section{Introduction}

In this project, we first review considerations for designing microRNA
(miRNA) profiling experiments that have appeared in recent literature
and then  explain and reproduce some of the statistical inferential
techniques proposed in \cite{witten2010ultra} for working with the
high-throughput data extracted from these experiments. In addition to
supplying \texttt{R} code within the text of this paper, all
preprocessed data and \texttt{R} analysis can be downloaded from
\href{http://github.com/krisrs1128}{http://github.com/krisrs1128}.

miRNAs are small ($\sim$22 nucleotide) noncoding RNAs that
facilitate regulation of gene expression through translational
inhibition \cite{kong2009strategies, pritchard2012microrna}. Further,
though it is estimated that there are only several thousand human
miRNAs (there are 2042 mature human miRNA sequences indexed in
\href{http://www.mirbase.org/cgi-bin/browse.pl?org=hsa}{mirBASE} as of
June 2013), they control expression patterns many coding genes; of
particular interest, in the last decade, it has been found
that miRNAs regulate transitions in plant and animal development,
including tissue differentiation \cite{ohnishi2010small,
  zhou2007mir, grosshans2005temporal}. Further, there are relevant
links between the study of miRNA function and research involving stem
cells and cancer \cite{alvarez2005microrna,   calin2006microrna,
  volinia2006microrna}.

Hence, the quantification of miRNAs expression levels between
diverse biological samples has become an important activity both for
fundamental biology, especially work to understand the mechanisms of
gene regulation, and medical research, such as the
identification of disease biomarkers. The diverse methodologies
developed towards these ends are referred to as miRNA profiling. There
is no standard, one-size-fits-all approach to the design of profiling
experiments \cite{kong2009strategies, pritchard2012microrna}. However, the
assortment of profiling techniques share the same overall pipeline:
RNA extraction, quantification and quality control, and profiling
\cite{pritchard2012microrna}. The specific choices at stages in this
pipeline, along with the research and sample considerations that guide
these choices, are the focus of section \ref{sec:pipeline}. In section
\ref{sec:inference}, we transition to the problem of inferring
interesting patterns from profiling data collected from RNA-seq
profiling platforms. In particular, we focus on explaining and
illustrating the high-dimensional statistical learning machinery found
useful in \cite{witten2010ultra}.

\section{miRNA profiling pipeline and considerations}
\label{sec:pipeline}

\subsection{RNA extraction and sample types}

The first step of any miRNA profiling experiment is the collection of
RNA for subsequent statistical inspection. miRNAs can be extracted
from a number of sample types, including cell lines, fresh tissues,
FACS, FFPE tissues, plasma, serum, and urine
\cite{pritchard2012microrna}. Cell lines and fresh tissues typically
have the highest yield, but can be more difficult to prepare; body
fluids typically have the smallest yield, but are easy to collect.
Fixed tissues can be convenient to work with, and it has been reported
that, unlike the longer mRNAs, miRNAs are stable in fixed tissues.

Once a sample type has been selected, standard RNA extraction
procedures can be applied, though they are typically modified to
promote enrichment of small RNAs. A number of \href{https://www.google.com/search?q=miRNA+isolation+kit&oq=miRNA+isolation+kit&aqs=chrome.0.57j0l3j62l2.3401j0&sourceid=chrome&ie=UTF-8}{commercially
  available
  kits} are available to isolate miRNAs; these are based on the
appropriate chemical extraction protocols. To increase representation
of small RNAs, size fractionation is common. However, this step is
not necessary, as profiling technologies, briefly reviewed in section
\ref{sec:profiling_tech}, are increasingly capable of distinguishing
miRNAs from other RNAs. Indeed, most profilers are capable of
resolving the smaller mature miRNAs from the larger precursor and
primary transcript miRNAs, despite the overlaps in base pairs
\cite{pritchard2012microrna, kong2009strategies}.

\subsection{Quantification and quality assurance}

Quantification is generally achieved through spectrophotometry or
automated capillary electrophoresis. Certain well-characterized miRNAs
can be taken as endogenous controls. Furthermore, standardization can
be achieved by spiking in control miRNAs. For example, for miRNA
extracted from body fluids, miRNAs from \textit{C. elegans} have been
artificially introduced in known amounts, and the measured values for
these controls guide normalization \cite{kroh2010analysis}. These
methods facilitate the reproducibility of study findings by separating
true biological signal from variations emerging from assay preparation
\cite{mestdagh2009novel, peltier2008normalization}.

\subsection{Profiling technologies}
\label{sec:profiling_tech}

miRNA profiling technologies provide measurements of the relative
abundances of miRNAs between different samples.
Two main approaches have emerged: direct hybridization without sample
amplification and variations of PCR-like amplification
\cite{kong2009strategies}.

The most common direct hybridization method employs oligo-microarrays.
Modifications have been designed to differentiate mature miRNAs from
their primary transcript and hairpin precursor miRNAs
\cite{liu2008expression}. These methods require larger initial total
RNA samples; however, since amplification is avoided, the measurements
obtained tend to be more stable and robust to variations in extract
preparation \cite{kong2009strategies}. Further, since many labs
already have microarray equipment, this approach to miRNA profiling is
more generally accessible.

As alternatives to direct hybridization, quantitative RNA sequencing
PCR (qRT-PCR) and RNA sequencing (RNA-seq) have been employed
\cite{kong2009strategies, pritchard2012microrna}. Since these methods
involve first amplifying the miRNA content in samples, it is possible
to use then with samples with smaller RNA content, such as plasmas or
needle-biopsies. In qRT-PCR, total RNA undergoes a modified reverse
transcriptase (RT) reaction to produce cDNA where, instead of the
standard poly-A reverse primers, a miRNA-specific reverse primer is
utilized. In the PCR step, mature miRNA-specific forwards primers are
introduced to selectively amplify miRNA sequences. As a consequence,
qRT-PCR can only be used to quantify abundances of known miRNAs; it
cannot identify novel miRNAs.

Conversely RNA-seq methods \emph{can} identify novel miRNAs; further,
they finely resolve between similar miRNA sequences
\cite{pritchard2012microrna}. The specific approaches differ between
platforms, but they all involve both hybridization and selective
amplification. For example, in miRAGE (miRNA serial analysis of gene
expression), magnetic linkers added to enriched small RNAs, whose
sequences are then reverse transcribed into cDNA. This cDNA is then
amplified by PCR; however, the PCR product is filtered for the
magnetic linkers. Only small RNA remains, which is then sequenced.
This sequencing step allows for identification of novel miRNAs.

\section{Statistical inference}
\label{sec:inference}

Section \ref{sec:pipeline} describes the overall workflow and
options possible for performing a miRNA profiling study. This overall
pipeline has been the subject of a number of recent reviews, including
\cite{pritchard2012microrna, kong2009strategies} cited above. Once
data is appropriately collected, focus shifts to performing valid and
powerful statistical inference. In this section, we review the
high-dimensional statistical learning methods described in
\cite{witten2010ultra} and reproduce their analysis using the data
available in their
\href{http://www.biomedcentral.com/1741-7007/8/58/additional}{additional
  files}. Since the measurements in \cite{witten2010ultra} was
collected from the RNA-seq profiling approach of
\ref{sec:profiling_tech}, our discussion will be tailored to
data collected through such amplification and sequencing. Their
experiment was designed to identify miRNAs (known and
novel) that are differentially abundant in normal and cervical cancer
tumor tissue.

\subsection{Preliminary observations}

Before applying specific inferential methods, it is worth previewing
the miRNA sequencing data. We read in the data below.


\begin{spacing}{1}
\begin{knitrout}
\definecolor{shadecolor}{rgb}{0.969, 0.969, 0.969}\color{fgcolor}\begin{kframe}
\begin{alltt}
X <- \hlfunctioncall{read.csv}(\hlstring{"miExpress2.csv"}, header = T)
\hlfunctioncall{rownames}(X) <- X[, 1]
X <- X[, -1]
nGenes <- \hlfunctioncall{nrow}(X)
nSamples <- \hlfunctioncall{ncol}(X)
X <- X^(1/3)  \hlcomment{# Cube-root transform}

sample.type <- \hlfunctioncall{substr}(\hlfunctioncall{colnames}(X), start = 6, stop = 6)
sample.type[59:60] <- \hlfunctioncall{c}(\hlstring{"N"}, \hlstring{"T"})  # two come from second run on same individual
sample.type <- \hlfunctioncall{factor}(sample.type)

nGenes
\end{alltt}
\begin{verbatim}
[1] 714
\end{verbatim}
\begin{alltt}
nSamples
\end{alltt}
\begin{verbatim}
[1] 60
\end{verbatim}
\begin{alltt}

X[1:5, 1:4]  \hlcomment{# Preview first five miRNAs, first four samples}
\end{alltt}
\begin{verbatim}
        G547.N G547.T G576.N G576.T
let-7a   9.528 14.952  9.322 17.088
let-7a*  1.442  2.289  2.289  4.626
let-7b   9.916 16.156 14.078 34.654
let-7b*  2.466  2.000  2.621  5.290
let-7c   9.390  8.036 10.775 16.708
\end{verbatim}
\begin{alltt}

\hlfunctioncall{summary}(sample.type)  \hlcomment{# Number of normal and tumor samples.}
\end{alltt}
\begin{verbatim}
 N  T 
30 30 
\end{verbatim}
\end{kframe}
\end{knitrout}

\end{spacing}


Observe that we work with expression of 714 miRNAs in paired normal
and cancerous tissues over 60 individuals, with 2
sequencing runs (those for G699N and G761T). Since the data is
right-skewed (a few measurements are very large, and most are close
to zero), a cube-root transformation is applied.

Figure \ref{fig:hist_counts} displays overall distribution of counts
between samples and miRNAs. The plot ``Counts per Sample'' shows the
total number of (cube-rooted) counts in different samples, color coded
according to whether the samples are normal or cancer tissue. The plot
``Counts per miRNA'' shows the cube rooted counts per miRNA over all
samples. Even after cube rooting, the distribution is right-skewed.
The plot ``Gene by Sample Counts'' displays the same information
without aggregating over either samples or miRNAs -- it shows the
distribution of individual elements in the data matrix.

\begin{spacing}{1}
\begin{figure}
\begin{knitrout}
\definecolor{shadecolor}{rgb}{0.969, 0.969, 0.969}\color{fgcolor}\begin{kframe}
\begin{alltt}
\hlfunctioncall{library}(ggplot2)
\hlfunctioncall{library}(reshape2)
mX <- \hlfunctioncall{melt}(X)
mX$sample.type <- \hlfunctioncall{rep}(sample.type, each = nGenes)
\hlfunctioncall{colnames}(mX) <- \hlfunctioncall{c}(\hlstring{"miRNA"}, \hlstring{"sample"}, \hlstring{"count"}, \hlstring{"sample.type"})
p1 <- \hlfunctioncall{qplot}(x = \hlfunctioncall{colSums}(X), geom = \hlstring{"histogram"}, fill = sample.type) + \hlfunctioncall{ggtitle}(\hlstring{"Counts per Sample"}) + 
    \hlfunctioncall{scale_x_continuous}(\hlstring{"Cube root of Sample Counts"}) + \hlfunctioncall{scale_y_continuous}(\hlstring{"Frequency"})
p2 <- \hlfunctioncall{ggplot}(mX) + \hlfunctioncall{geom_histogram}(\hlfunctioncall{aes}(x = count, fill = sample.type)) + \hlfunctioncall{ggtitle}(\hlstring{"Gene by Sample Counts"}) + 
    \hlfunctioncall{scale_y_continuous}(\hlstring{"Frequency"})
p3 <- \hlfunctioncall{qplot}(x = \hlfunctioncall{rowSums}(X), geom = \hlstring{"histogram"}) + \hlfunctioncall{ggtitle}(\hlstring{"Counts per miRNA"}) + 
    \hlfunctioncall{scale_x_continuous}(\hlstring{"Cube root of Sample Counts"}) + \hlfunctioncall{scale_y_continuous}(\hlstring{"Frequency"})
\hlfunctioncall{multiplot}(p1, p2, p3, cols = 2)
\end{alltt}
\end{kframe}
\includegraphics[width=\maxwidth]{figure/histogram_counts} 

\end{knitrout}

\caption{Histograms of sample aggregated, miRNA aggregated, and
  unaggregated counts in the data provided by \cite{witten2010ultra}.}
\label{fig:hist_counts}
\end{figure}
\end{spacing}

\subsection{Unsupervised methods}
\label{sec:unsupervised}

The first class of methods relevant to our inference are called
unsupervised methods. In these methods, the relation between samples
is characterized without using the labels associated with them (in
this case, the label is whether or not the sample comes from normal or
cancerous tissue). The nature of the characterization depends on the
specific method, but often involves finding appropriate
lower-dimensional representations of or significant separations within
samples.

If these algorithms, despite being trained independently of
class labels, nonetheless identify separations corresponding to
class labels, then we can deduce that the most important variations in
the data are associated with these labels. We find this to be the case
with our miRNA data, indicating that collected miRNA expression levels
exhibit strong associations with the presence or cervical cancer. It
is in the domain of the supervised methods of section
\ref{sec:supervised_methods} to identify which of the 714
are responsible for these associations.

\subsection{Principal components analysis}

A first thought is to apply principal components analysis (PCA).
Conceptually, if we think about each sample as a point in
714-dimensional space (each coordinate corresponding to the
expression level of a single miRNA), then PCA finds a lower
$d$-dimensional plane (we take $d = 2$ for visualization purposes)
that best approximates all 60 sample points in our data.
Mathematically, it can be demonstrated that this plane also maximizes
the variation of the projection of points onto a $d$-dimensional
plane, so we can interpret the principal components as the linear
combinations of covariates that explain the most variation in the
data.

\begin{spacing}{1}
\begin{knitrout}
\definecolor{shadecolor}{rgb}{0.969, 0.969, 0.969}\color{fgcolor}\begin{kframe}
\begin{alltt}
\hlcomment{# Code for PCA}
pc.X <- \hlfunctioncall{function}(X) \{
    X.tilde <- \hlfunctioncall{scale}(X, center = T, scale = F)
    svdX <- \hlfunctioncall{svd}(X.tilde)
    pca.X <- svdX$u %*% \hlfunctioncall{diag}(svdX$d)
    pca.approx <- pca.X[, 1:2]
    \hlfunctioncall{return}(pca.approx)
\}

pca.labeled <- \hlfunctioncall{data.frame}(\hlfunctioncall{pc.X}(\hlfunctioncall{t}(X)), \hlfunctioncall{pc.X}(1 - \hlfunctioncall{cor}(X)), label = \hlfunctioncall{colnames}(X))
\hlfunctioncall{colnames}(pca.labeled) <- \hlfunctioncall{c}(\hlstring{"PC1"}, \hlstring{"PC2"}, \hlstring{"PC1.cor"}, \hlstring{"PC2.cor"}, \hlstring{"label"})
\end{alltt}
\end{kframe}
\end{knitrout}

\end{spacing}

The most direct approach
to PCA for our miRNA data is to work
with the cube-rooted data visualized before. This is displayed on the
left of figure \ref{fig:pca}. Alternatively, we can construct a
distance based on the correlations between samples. We then find the
best 2-dimensional approximation for this data. The result is
displayed on the right in \ref{fig:pca}.

Notice that both approaches effectively separate samples obtained
from tumorous and normal tissues, despite being constructed without
any label of the state of samples. This is corroborates the
claim above that the miRNA data supplied contains strong associations
between expression level and tumor-state.

\begin{spacing}{1}
  \begin{figure}
\begin{knitrout}
\definecolor{shadecolor}{rgb}{0.969, 0.969, 0.969}\color{fgcolor}\begin{kframe}
\begin{alltt}
pca.plot <- \hlfunctioncall{ggplot}(pca.labeled) + \hlfunctioncall{geom_text}(\hlfunctioncall{aes}(x = PC1, y = PC2, label = label, 
    col = sample.type)) + \hlfunctioncall{ggtitle}(\hlstring{"PCA from rooted data"})
pca.plot.cor <- \hlfunctioncall{ggplot}(pca.labeled) + \hlfunctioncall{geom_text}(\hlfunctioncall{aes}(x = PC1.cor, y = PC2.cor, 
    label = label, col = sample.type)) + \hlfunctioncall{ggtitle}(\hlstring{"PCA from correlation distance"})

\hlfunctioncall{multiplot}(pca.plot, pca.plot.cor)
\end{alltt}
\end{kframe}
\includegraphics[width=\maxwidth]{figure/pca_plot} 

\end{knitrout}

\caption{2-dimensional approximation obtained by PCA. Notice that the
  sample labels are distinguished even though the algorithm does not
  use them in its development. The correlation approach yields less
  skewed groups.}
\label{fig:pca}
\end{figure}
\end{spacing}

\subsection{Hierarchical clustering}

The second unsupervised method we consider is hierarchical clustering.
Given pairwise distances between samples, hierarchical
clustering recursively pairs closest samples, resulting in a tree
allowing visualization of the distances between all samples. In a
sense, it serves as a function to map local pairwise distance
information into a global representation of the relationship between
all samples.

More precisely, given all pairwise distances between samples, we first
merge the two points closest to each other, calling this a cluster. At
the next iteration, we recompute pairwise distances, where the
distance of points to this new cluster is defined as the distance to the
farthest point in this cluster. This procedure is repeated, where
the distance between two clusters is defined to be the maximum of the
pairwise distances between points contained within either cluster. The
sequences of merges is displayed in a tree. The heights of the merges
in the tree are the distances between the clusters when they are
merged. Notice that no where in this algorithm is a labeling of points
used.

For our application, we define the distance between two samples
with estimated miRNA abundances correlation $\hat{\rho}$ to be $1 -
\left|\hat{\rho}\right|$. This is intuitively reasonable: points with
perfect correlation should have zero distance, while
if they are uncorrelated, they should have maximum distance to each
other. The result of the clustering is displayed in figure
\ref{fig:hclust}. Amazingly, with the exception of samples
G428\_T and G701\_T, the clustering perfectly separates the normal
samples from the tumor samples. Again, this corroborates the idea that
miRNA abundance effectively distinguishes between cancer and normal
samples.
\begin{spacing}{1}
  \begin{figure}
\begin{knitrout}
\definecolor{shadecolor}{rgb}{0.969, 0.969, 0.969}\color{fgcolor}\begin{kframe}
\begin{alltt}
\hlfunctioncall{source}(\hlstring{"http://addictedtor.free.fr/packages/A2R/lastVersion/R/code.R"})  # Plotting Package
miRNA.clust <- \hlfunctioncall{hclust}(\hlfunctioncall{as.dist}(1 - \hlfunctioncall{cor}(X)))
\hlfunctioncall{A2Rplot}(miRNA.clust, k = 3, col.down = \hlfunctioncall{c}(\hlstring{"#FF6B6B"}, \hlstring{"#4ECDC4"}, \hlstring{"#556270"}), main = \hlstring{"Hierchical Clustering of Patients using miRNA Counts"})
\end{alltt}
\end{kframe}
\includegraphics[width=\maxwidth]{figure/hclustPlot} 

\end{knitrout}

\caption{Hierarchical clustering of samples using correlation
  obtained using miRNA expression levels.}
\label{fig:hclust}
\end{figure}
\end{spacing}

\subsection{Supervised methods}
\label{sec:supervised_methods}

Next, we consider the application of supervised learning methods. The
distinguishing feature of these algorithms is that they need to
be trained with sample labels. These methods can be trained to perform
classification and identify differential expression of miRNAs between
tumor and normal and tissue samples. Hence, while the methods of
section \ref{sec:unsupervised} were able to identify strong
associations between global miRNA expression and cancer state, the
methods below will be able to resolve the particular miRNAs that
contribute to these differences.

\subsubsection{NSC classification}

Again, we consider each sample to be a point in
714-dimensional space where each coordinate is the
expression level of a particular miRNA. The goal is to train a
classifier such that, given a vector of expression levels of these
miRNAs for a new tissue, we can decide whether the sample came from a
cervical cancer or normal tissue. Further, we would like the
classifier to output which are the differentially expressed miRNAs
that contribute to the classification. An approach designed
specifically for this kind of high-dimensional setting is nearest
shrunken centroids (NSC) \cite{tibshirani2003class,
  hastie2005elements}.

Heuristically, the NSC algorithm is a version of linear
discriminant analysis regularized to be appropriate for the
high-dimension of samples. Specifically, given a new sample, we can
compute the distance to the known average expression levels for the
tumorous and normal tissues in our training data (the distance is
taken with respect to the estimated variances of individual miRNA
expression levels).

The naive approach would be to classify the new sample to the closest
of these two averages. However, this presumes that every miRNA
contributes to the differences between normal and tumorous samples.
We don't expect this to be the case, so we soft-threshold the
averages until only the miRNAs contributing the most to the
classification are included. We then classify the new sample to the
closest of these thresholded averages.

Of course, we don't know in advance the appropriate level to threshold
these averages. Hence, we apply cross-validation. The idea is to fix a
thresholding level, train the classifier on a subset of samples, and
calculate the error rate of the classification of the remaining
samples. A range of thresholding levels is considered, and that
yielding the minimum error rate is selected. This classifier can now
be used on entirely new data.

We apply the \texttt{R} package
\href{http://cran.r-project.org/web/packages/pamr/index.html}{\texttt{pamr}}
to perform NCS classification on our miRNA data. We display the
cross-validation results in figure \ref{fig:cv}, the expression levels
of miRNAs used in the classifier in figure \ref{fig:geneplot}, and the
corresponding centroids in \ref{fig:centroids}. The cross-validation
results are typical: the under and overthresholded models have higher
misclassification rates. We choose the threshold 3.3021,
because it is the largest (among those tested) that still yields a
minimum cross-validation error; intuitively, this is the simplest
model achieving minimum estimated average test error, which we find to
be about 0.1667 (compared to 0.5 for random
classifications).

The classifier uses 15 of the 714 total
miRNAs. From figures \ref{fig:geneplot} and \ref{fig:centroids}, we
find that all of these miRNAs, with the exception of miR-205 are less
expressed in tumor samples than in the normal tissues. We can now look up
these miRNAs on miRBase. For example,
\href{http://www.mirbase.org/cgi-bin/mirna_entry.pl?acc=MI0000446}{miR-125b}
has been found to be associated with cell differentiation, and
\href{http://www.mirbase.org/cgi-bin/mirna_entry.pl?acc=MI0000459}{miR-143}
has been implicated in both cardiac morphogenesis and certain types of
cancer.

\begin{spacing}{1}
\begin{knitrout}
\definecolor{shadecolor}{rgb}{0.969, 0.969, 0.969}\color{fgcolor}\begin{kframe}
\begin{alltt}
\hlfunctioncall{library}(\hlstring{"pamr"})
nsc.train.data <- \hlfunctioncall{list}(x = X, y = sample.type, genenames = \hlfunctioncall{rownames}(X))
training.results <- \hlfunctioncall{pamr.train}(nsc.train.data)
\hlfunctioncall{set.seed}(\hlstring{"672013"})  # Cross-validation includes randomness
nsc.cv.fit <- \hlfunctioncall{pamr.cv}(training.results, nsc.train.data)
cv.thresh.ix <- \hlfunctioncall{max}(\hlfunctioncall{which}(nsc.cv.fit$error == \hlfunctioncall{min}(nsc.cv.fit$error)))
cv.thresh <- nsc.cv.fit$threshold[cv.thresh.ix]
nMisclassified <- nsc.cv.fit$error[cv.thresh.ix] * 60
nGenesUsed <- nsc.cv.fit$size[cv.thresh.ix]
\end{alltt}
\end{kframe}
\end{knitrout}

\end{spacing}

\begin{spacing}{1}
\begin{figure}
\begin{knitrout}
\definecolor{shadecolor}{rgb}{0.969, 0.969, 0.969}\color{fgcolor}\begin{kframe}
\begin{alltt}
\hlfunctioncall{pamr.plotcv}(nsc.cv.fit)
\end{alltt}
\end{kframe}
\includegraphics[width=\maxwidth]{figure/plotcv} 

\end{knitrout}

\caption{Misclassification rates estimated for different values of the
  thresholding parameter, using cross-validation.}
\label{fig:cv}
\end{figure}
\end{spacing}

\begin{spacing}{1}
\begin{figure}
\begin{knitrout}
\definecolor{shadecolor}{rgb}{0.969, 0.969, 0.969}\color{fgcolor}\begin{kframe}
\begin{alltt}
\hlfunctioncall{pamr.geneplot}(training.results, nsc.train.data, threshold = cv.thresh)
\end{alltt}
\end{kframe}
\includegraphics[width=\maxwidth]{figure/geneplot} 

\end{knitrout}

\caption{microRNA expression levels in tumor and normal samples.
  Reading from left to right, top to bottom, the indices in the titles
  correspond to the miRNA names, read top to bottom,  in figure
  \ref{fig:centroids}. Green gives expression levels in tumorous
  samples while red gives expression levels in normal tissues.}
\label{fig:geneplot}
\end{figure}
\end{spacing}

\begin{spacing}{1}
\begin{figure}
\begin{knitrout}
\definecolor{shadecolor}{rgb}{0.969, 0.969, 0.969}\color{fgcolor}\begin{kframe}
\begin{alltt}
\hlfunctioncall{pamr.plotcen}(training.results, nsc.train.data, threshold = cv.thresh)
\end{alltt}
\end{kframe}
\includegraphics[width=\maxwidth]{figure/centroids} 

\end{knitrout}

\caption{These are the centroids employed in classification after a
  threshold has been chosen. The larger the centroid coordinate for a
  particular RNA, the stronger the association.}
\label{fig:centroids}
\end{figure}
\end{spacing}

\subsubsection{Testing differential expression}

Finally, we can formally test for the differential expression of
miRNAs between samples and estimate the associated false discovery
rate (FDR). We describe a simple approach; a more sophisticated technique
can be read in \cite{witten2010ultra}.

The basic idea is to perform a $t$-test for differential
expression of each miRNA. Due to the multiple testing, we cannot
directly use the analytic $p$-values for the $t$-test to determine
which results are significant at any prespecified level. Instead, we
nonparametrically estimate the FDR via permutations.

Recall that the FDR is defined as the expected
proportion of false positives among all hypotheses rejected. To
estimate this quantity, we observe that under the null hypothesis of
equal miRNA expression between tumor and normal tissue classes, we can
permute class labels without changing the distribution of test
statistics. Hence, to estimate the FDR, assuming a
fixed number of rejected hypotheses, we can permute the class labels
to calculate the null distribution of test statistics. We can compare
the test statistics calculated on our real data with this null
distribution; the FDR associated with $k$ rejections
is then the proportion of test statistics in the null distribution
larger than the statistic $t_{\left(k\right)}$ (the $k^{th}$ largest
statistic in our tests on true data).

The estimated FDRs are displayed in figure
\ref{fig:fdr}. 91 miRNAs are deemed
significant at an FDR level of 0.05; they are stored in the object
\texttt{significant.genes}. The first few are displayed in the code
chunk below.

\begin{spacing}{1}
\begin{knitrout}
\definecolor{shadecolor}{rgb}{0.969, 0.969, 0.969}\color{fgcolor}\begin{kframe}
\begin{alltt}
t.test.miRNA <- \hlfunctioncall{function}(X, sample.type) \{
\hlcomment{    # Normalize each individual's mean miRNA counts}
    X.tilde <- \hlfunctioncall{apply}(X, MARGIN = 2, FUN = \hlfunctioncall{function}(v) \{
        v/(\hlfunctioncall{mean}(v))
    \})
\hlcomment{    # Two sample t-test}
    t.statistics <- \hlfunctioncall{apply}(X.tilde, MARGIN = 1, FUN = \hlfunctioncall{function}(gene) \hlfunctioncall{t.test}(gene ~ 
        sample.type)$statistic)
    sort.genes <- \hlfunctioncall{sort}(t.statistics, decreasing = T)
    
    \hlfunctioncall{return}(sort.genes)
\}

perm.statistics <- \hlfunctioncall{function}(X, sample.type, n.perm = 100) \{
    nGenes <- \hlfunctioncall{nrow}(X)
    t.statistics.perm <- \hlfunctioncall{matrix}(nrow = n.perm, ncol = nGenes)
    \hlfunctioncall{for} (i in 1:n.perm) \{
\hlcomment{        # Generate random permutations}
        sample.type.perm <- \hlfunctioncall{sample}(sample.type)
        t.statistics.perm[i, ] <- \hlfunctioncall{t.test.miRNA}(X, sample.type.perm)
    \}
    \hlfunctioncall{return}(t.statistics.perm)
\}

fdr.rates <- \hlfunctioncall{function}(true.statistics, perm.stats) \{
    nGenes <- \hlfunctioncall{length}(true.statistics)
    fdr.rate <- \hlfunctioncall{vector}(length = nGenes)
    \hlfunctioncall{for} (i in 1:nGenes) \{
        thresh <- true.statistics[i]
        fdr.rate[i] <- \hlfunctioncall{length}(\hlfunctioncall{which}(permuted.stats >= thresh))/(\hlfunctioncall{nrow}(perm.stats) * 
            i)
    \}
    \hlfunctioncall{return}(fdr.rate)
\}

genes.t.test <- \hlfunctioncall{t.test.miRNA}(X, sample.type)
permuted.stats <- \hlfunctioncall{perm.statistics}(X, sample.type)
t.test.fdr <- \hlfunctioncall{fdr.rates}(genes.t.test, permuted.stats)

n.rejected.05level <- \hlfunctioncall{max}(\hlfunctioncall{which}(t.test.fdr <= 0.05))
significant.genes <- genes.t.test[1:n.rejected.05level]
\hlfunctioncall{head}(significant.genes)
\end{alltt}
\begin{verbatim}
##     miR-143    miR-125b    miR-145*    miR-10b* miR-125a-5p     miR-204 
##       8.424       8.337       7.598       7.132       7.100       7.076
\end{verbatim}
\end{kframe}
\end{knitrout}

\end{spacing}

\begin{spacing}{1}
  \begin{figure}
\begin{knitrout}
\definecolor{shadecolor}{rgb}{0.969, 0.969, 0.969}\color{fgcolor}\begin{kframe}
\begin{alltt}
\hlfunctioncall{qplot}(x = 1:200, t.test.fdr[1:200]) + \hlfunctioncall{scale_x_continuous}(\hlstring{"Number of Significant miRNAs Declared"}) + 
    \hlfunctioncall{scale_y_continuous}(\hlstring{"FDR"}) + \hlfunctioncall{ggtitle}(\hlstring{"Estimated FDR for t-tests"}) + \hlfunctioncall{geom_line}(y = 0.05, 
    col = \hlstring{"red"})
\end{alltt}
\end{kframe}
\includegraphics[width=\maxwidth]{figure/miRNA_fdr} 

\end{knitrout}

\caption{FDR.}
\label{fig:fdr}
\end{figure}
\end{spacing}

\section{Conclusion}

In this report, we explored current techniques in
experimental preparation and statistical analysis of miRNA profiling
data. We found that there is no universal approach to collecting
miRNA profiling data, but that the necessary choices can be guided by
sample and research considerations. Further, in our reproduction of
statistical analysis of RNA-seq pipeline profiling data, we found
strong associations between specific miRNA expression levels and the
tumor-status of tissues; indeed, associations were detected even via
unsupervised methods.

More broadly, this project has given a chance to work in the
intersection between modern genetics and statistics. We have seen how
an array of creative experimental and statistical techniques have been
designed to both deepen our understanding of miRNAs and the regulation of
gene expression as well as provide potentially powerful biomarkers for
diseases. As personalized genomics and medicine
continues to expand and the associated data and research questions
grow in complexity, we anticipate challenging problems whose
solutions can be informed by the themes that we have encountered in
this report and which present rich research and healthcare
opportunities.

\bibliographystyle{plain}
\bibliography{Biochem158_project}

\end{document}
